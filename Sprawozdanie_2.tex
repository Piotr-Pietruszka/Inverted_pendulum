\documentclass{article}

%\usepackage[T1]{fontenc}
%\usepackage[utf8]{inputenc}
%\usepackage{polski}
\usepackage[utf8]{inputenc}
\usepackage{polski}
\usepackage[polish]{babel}
\usepackage{mathtools}
\usepackage[thinc]{esdiff}
\usepackage{graphicx} 

\usepackage{float}

\usepackage{geometry}

\begin{document}
\newgeometry{tmargin =3cm, bmargin=3cm, lmargin=3cm, rmargin=3cm}
\begin{tabular}{|c|c|c|}
\hline 
\multicolumn{3}{|c|}{\huge Sprawozdanie } \\ 
\hline 
\multicolumn{3}{|c|}{\LARGE Projekt 2 NMTS - Odwrócone wahadło} \\ 
\hline 
\Large Przygotowali: &\Large Piotr Pietruszka 171842 &\Large Marcin Wankiewicz 172118  \\ 
\hline 
\Large Kierunek: ACiR  \\ 
\hline 
 
\end{tabular} 

\section{Zlinearyzowany model stanowy}

Równania różniczkowe opisujące dynamikę układu przedstawiono w: \ref{eq:dyn}.
\begin{equation}\label{eq:dyn}
 \begin{array}{l}
  (M+m) \ddot{y} + b\dot{y} = F + mL\theta^2 \sin\theta - mL\ddot{\theta}\cos\theta \\
  (I + mL^2)\ddot{\theta} = mgL\sin\theta - mL\ddot{y}\cos\theta
 \end{array}
\end{equation}
W celu linearyzacji wokół punktu $\theta=0$, dla równań \ref{eq:dyn} użyto następujące przybliżenia: \ref{eq:appr}.
\begin{equation}\label{eq:appr}
 \begin{array}{l}
  \cos\theta = 1,\;  \sin\theta=\theta, \; \dot{\theta}^2 = 0
 \end{array}
\end{equation}

W wyniku otrzymano model stanowy (macierze zgodnie ze standardowymi oznaczeniami): \ref{eq:ss}, gdzie wektor stanu $\textbf{x}$ jest zdefiniowany w \ref{eq:sv}, a wejściem jest siła $F$ działająca na wózek.

\begin{equation}\label{eq:ss}
 \begin{array}{l}
  \mathbf{A} = \begin{bmatrix}  0 & 1 & 0 & 0 \\
  							   0 & -\frac{(I+mL^2)b}{I(M+m) + MmL^2} & -\frac{m^2gL^2}{I(M+m) + MmL^2} & 0 \\
  							   0 & 0 & 0 & 1 \\
  							   0 & \frac{bmL}{I(M+m) + MmL^2} & \frac{gmL(M+m)}{I(M+m) + MmL^2} & 0 \\ 
  			   \end{bmatrix} \\ \\
  			   
  \mathbf{B} = \begin{bmatrix} 0 \\ \frac{I + mL^2}{I(M+m) + MmL^2} \\ 0 \\ -\frac{(mL}{I(M+m) + MmL^2} \end{bmatrix} \\ \\
  
  \mathbf{C} = \begin{bmatrix}  1 & 0 & 0 & 0 \\
  							   0 & 1 & 0 & 0 \\
  							   0 & 0 & 1 & 0 \\
  							   0 & 0 & 0 & 1 \\ 
  			   \end{bmatrix} \\ \\
  
  \mathbf{D} = \begin{bmatrix} 0 \end{bmatrix} \\
\end{array}
\end{equation}

\begin{equation}\label{eq:sv}
 \begin{array}{l}
	\mathbf{x} = 
	\begin{bmatrix} 
	  y \\ \dot{y} \\ \theta \\ \dot{\theta} 
	\end{bmatrix} 
 \end{array}
\end{equation}
Po podstawieniu wartości liczbowych: \ref{eq:ss_val}.

\begin{equation}\label{eq:ss_val}
 \begin{array}{l}
  \mathbf{A} = \begin{bmatrix}  0 & 1 & 0 & 0 \\
  							   0 & -1.16896918\cdot10^{-2} & -2.81381031\cdot10^{-1} & 0 \\
  							  0 & 0 & 0 & 1 \\
  							   0 & 3.18809777\cdot10^{-2} & 2.75128119\cdot10^{1} & 0 \\ 
  			   \end{bmatrix} \\ \\
  			   
  \mathbf{B} = \begin{bmatrix} 0 \\ 0.11689692 \\ 0 \\ -0.31880978 \end{bmatrix} \\ \\
  
  \mathbf{C} = \begin{bmatrix} 1 & 0 & 0 & 0 \\
  							   0 & 1 & 0 & 0 \\
  							   0 & 0 & 1 & 0 \\
  							   0 & 0 & 0 & 1 \\ 
  			   \end{bmatrix} \\ \\
  
  \mathbf{D} = \begin{bmatrix} 0 \end{bmatrix} \\
\end{array}
\end{equation}

\section{Sprzężenie od stanu}

% Tu ja widzę opis uzyskania kontrolera tym Ackermanem

% Nie wiem czy to zdanie dodać jeszcze się zastanowię
%Sterownik został zaimplementowany w programie, jako ...

% Tu może dodam jak wygląda model stanowy z kontrolerem (choć z niego bezpośrednio nigdzie nie korzystamy)

\section{LQR}












\section{Templates - do usuniecia}
\begin{equation}\label{eq:standard_ss_template}
 \begin{array}{l}
  \mathbf{A} = \begin{bmatrix}  1 & 1 \\
  							   1 & 1 
  			   \end{bmatrix} \\ \\
  \mathbf{B} = \begin{bmatrix} 1 \\ 1 \end{bmatrix} \\ \\
  \mathbf{C} = \begin{bmatrix} 1 & 1 \end{bmatrix} \\ \\
  \mathbf{D} = \begin{bmatrix} 0 \end{bmatrix} \\
\end{array}
\end{equation}



\end{document}


